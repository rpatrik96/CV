%%%%%%%%%%%%%%%%%%%%%%%%%%%%%%%%%%%%%%%%%
% "ModernCV" CV and Cover Letter
% LaTeX Template
% Version 1.11 (19/6/14)
%
% This template has been downloaded from:
% http://www.LaTeXTemplates.com
%
% Original author:
% Xavier Danaux (xdanaux@gmail.com)
%
% License:
% CC BY-NC-SA 3.0 (http://creativecommons.org/licenses/by-nc-sa/3.0/)
%
% Important note:
% This template requires the moderncv.cls and .sty files to be in the same 
% directory as this .tex file. These files provide the resume style and themes 
% used for structuring the document.
%
%%%%%%%%%%%%%%%%%%%%%%%%%%%%%%%%%%%%%%%%%


%%%%%%%%%%%%%%%%%%%%%%%%%%%%%%%%%%%%%%
%	PACKAGES AND OTHER DOCUMENT CONFIGURATIONS
%%%%%%%%%%%%%%%%%%%%%%%%%%%%%%%%%%%%%%

\documentclass[11pt,a4paper,roman]{moderncv} % Font sizes: 10, 11, or 12; paper sizes: a4paper, letterpaper, a5paper, legalpaper, executivepaper or landscape; font families: sans or roman

%--------------------------------------------------------------------------------------
% Abbreviations
%--------------------------------------------------------------------------------------
\newcommand{\ie}{i.e.\@\xspace}
\newcommand{\Ie}{I.e.\@\xspace}
\newcommand{\eg}{e.g.\@\xspace}
\newcommand{\Eg}{E.g.\@\xspace}
\newcommand{\etal}{et al.\@\xspace}
\newcommand{\etc}{etc.\@\xspace}
\newcommand{\vs}{vs.\@\xspace}
\newcommand{\viz}{viz.\@\xspace} % videlicet
\newcommand{\cf}{cf.\@\xspace} % confer
\newcommand{\Cf}{Cf.\@\xspace}
\newcommand{\wrt}{w.r.t.\@\xspace} % with respect to
\newcommand{\wolog}{w.l.o.g.\@\xspace} % w/o loss of generality

% font loading
% for luatex and xetex, do not use inputenc and fontenc
% see https://tex.stackexchange.com/a/496643
\ifxetexorluatex
  \usepackage{fontspec}
  \usepackage{lmodern}
  \usepackage{unicode-math}
  \defaultfontfeatures{Ligatures=TeX}
    % fontspec allows you to use TTF/OTF fonts directly
    \setmainfont{MinionPro}[ 
    Extension = .otf,
    BoldFont = *-Bold,
    ItalicFont = *-It,
    UprightFont = *-Regular,
    ]
\else
  \usepackage[T1]{fontenc}
  \usepackage{lmodern}
\fi

\moderncvstyle{classic} % CV theme - options include: 'casual' (default), 'classic', 'oldstyle' and 'banking'
\moderncvcolor{customblue} % CV color - options include: 'blue' (default), 'orange', 'green', 'red', 'purple', 'grey' and 'black'

\usepackage{lipsum} % Used for inserting dummy 'Lorem ipsum' text into the template

\usepackage[nodayofweek]{datetime}
\usepackage{multicol} % Used for itemizing items in multiple columns

\usepackage[scale=0.75]{geometry} % Reduce document margins
%\setlength{\hintscolumnwidth}{3cm} % Uncomment to change the width of the dates column
%\setlength{\makecvtitlenamewidth}{10cm} % For the 'classic' style, uncomment to adjust the width of the space allocated to your name


\usepackage{xstring}
\usepackage[normalem]{ulem} 

\let\originalbibitem\bibitem
\def\bibitem#1#2\par{%
  \noexpandarg
  \originalbibitem{#1}
  \StrSubstitute{#2}{Patrik Reizinger}{{\color{customblue}{\textbf{Patrik Reizinger}}}}\par}



\usepackage{etoolbox}
\usepackage{ifthen}

\newtoggle{en}
\newcommand{\en}[2]{\iftoggle{en}{#1}{#2}}

\toggletrue{en}

\iftoggle{en}{}{\renewcommand\refname{Publikációk}}

\newdateformat{hundate}{\THEYEAR.\twodigit{\THEMONTH}.\twodigit{\THEDAY}.}



%%%%%%%%%%%%%%%%%%%%%%%%%%%%%%%%%%%%%%
%	FONT
%%%%%%%%%%%%%%%%%%%%%%%%%%%%%%%%%%%%%%


\usepackage[calcwidth]{titlesec}				
\usepackage{multicol}
\usepackage{multirow}
%CV Sections inspired by: 
%http://stefano.italians.nl/archives/26
\titleformat{\section}{\Large\scshape\raggedright\color{customblue}}{}{0em}{\titlerule[1pt]}[\vspace{-4.55ex}\rule{\titlewidth}{2.5pt}\vspace{1.75ex}]

\titlespacing{\section}{0pt}{10pt}{10pt}

%%%%%%%%%%%%%%%%%%%%%%%%%%%%%%%%%%%%%%
%	NAME AND CONTACT INFORMATION SECTION
%%%%%%%%%%%%%%%%%%%%%%%%%%%%%%%%%%%%%%

\firstname{\en{Patrik}{Reizinger}} % Your first name
\familyname{\en{Reizinger}{Patrik}} % Your last name

% All information in this block is optional, comment out any lines you don't need
% \title{Curriculum Vitae}
% \address{XY}{Z}


\extrainfo{
\href{mailto:patrik.reizinger@tuebingen.mpg.de}{\raisebox{-0.05\height}{\color{customblue}\faEnvelope} \ patrik.reizinger@tuebingen.mpg.de} \\
\href{tel:+36306137236}{\raisebox{-0.05\height}{\color{customblue}\faMobile} \ +36-30-613-7236} \\
\href{https://github.com/rpatrik96}{\raisebox{-0.05\height}{\color{customblue}\faGithub}\ rpatrik96}\\ 
\href{https://linkedin.com/in/patrik-reizinger}{\raisebox{-0.05\height}{\color{customblue}\faLinkedin}\ patrik-reizinger}\\
\href{https://rpatrik96.github.io}{\raisebox{-0.05\height}{\color{customblue}\faGlobe} \ rpatrik96.github.io} 
}

% \photo{rp_small.jpeg}

\usepackage{hyphenat}
\hyphenation{mi-nisz-té-ri-um}

%%%%%%%%%%%%%%%%%%%%%%%%%%%%%%%%%%%%%%

\begin{document}

\makecvtitle % Print the CV title









%%%%%%%%%%%%%%%%%%%%%%%%%%%%%%%%%%%%%%
\section{\en{Education}{Tanulmányok}}
%%%%%%%%%%%%%%%%%%%%%%%%%%%%%%%%%%%%%%



% \cventry{years}{degree/job title}{institution/employer}
% {localization}{optional: grade/...}
% {optional: comment/job description}

\cventry{2021--\en{2025}{jelenleg}}{Machine Learning Ph.D.}{International Max Planck Research School for Intelligent Systems/University of Tübingen/ELLIS}{Tübingen, \en{Germany}{Németország}}{}{\textbf{\en{Thesis}{Tézis}:} Causal Representation Learning\\ \textbf{\en{Supervisors}{Témavezetők}:} Wieland Brendel, \en{Ferenc Huszár}{Huszár Ferenc}, Matthias Bethge, Bernhard Schölkopf\\
\textbf{ELLIS \en{exchange at the}{csereprogram,} University of Cambridge:} 2022.10--2023.03.}

\cventry{2019--2021}{\en{Electrical Engineering M.Sc.}{Villamosmérnök M.Sc.}}{\en{Budapest University of Technology and Economics}{Budapesti Műszaki és Gazdaságtudományi Egyetem}}{Budapest, \en{Hungary}{Magyarország}}{GPA~5.0/5.0 (\en{valedictorian}{évfolyamelső})}{\textbf{\en{Thesis}{Diplomamunka}:} \en{Development of an Attitude Determination and Control System for CubeSats on LEO orbits}{Orientációmeghatározó és -szabályozó alrendszer fejlesztése LEO pályán működő kisműholdakhoz}\\
\textbf{\en{Supervisors}{Témavezetők}:} \en{Ferenc Vajda}{Vajda Ferenc}, \en{Márton Szemenyei}{Szemenyei Márton}\\
\textbf{\en{Extracurricular}{Kiegészítő programok}:} iMSc \en{}{tehetséggondozó} program \en{for talented students}{}}

\cventry{2015--2019}{\en{Electrical Engineering B.Sc.}{Villamosmérnök M.Sc.}}{\en{Budapest University of Technology and Economics}{Budapesti Műszaki és Gazdaságtudományi Egyetem}}{Budapest, \en{Hungary}{Magyarország}}{GPA~5.0/5.0 (\en{valedictorian}{évfolyamelső})}{\textbf{\en{Thesis}{Diplomamunka}:} \en{Development of a 3D input device for virtual working environments}{Periféria fejlesztése virtuális munkakörnyezetekhez}\\
\textbf{\en{Supervisors}{Témavezetők}:} \en{Ferenc Vajda}{Vajda Ferenc}, \en{Márton Szemenyei}{Szemenyei Márton}\\
\textbf{\en{Extracurricular}{Kiegészítő programok}:} \en{German language program in cooperation with the}{Német nyelvű mérnökképzés} Karlsruhe Institute of Technology\\
\textbf{\en{Exchange semester at the}{ERASMUS félév,} Karlsruhe Institute of Technology:} 2018.10--2019.02.}




%%%%%%%%%%%%%%%%%%%%%%%%%%%%%%%%%%%%%%
\section{\en{Experience}{Munkatapasztalat}}
%%%%%%%%%%%%%%%%%%%%%%%%%%%%%%%%%%%%%%
\cventry{2024.06--2024.09.}{Summer Research Intern}{Vector Institute}{Toronto, \en{Canada}{Kanada}}{}{\en{Hosted by Rahul G. Krishnan, working on causal representation learning and self-supervised methods}{}}
\cventry{2020.02--2021.01.}{Deep Learning Student Researcher}{\en{Budapest University of Technology and Economics}{Budapesti Műszaki és Gazdaságtudományi Egyetem}}{Budapest, \en{Hungary}{Magyarország}}{}{\en{Analyzed time series data with deep learning}{Idősoros adatok elemzése deep learning módszerekkel}}
\cventry{2019.02--2021.02.}{Control Engineering Intern}{C3S Electronics Development LLC}{Budapest, \en{Hungary}{Magyarország}}{}{\en{Developed and designed a CubeSat attitude determination and control system}{Orientációmeghatározó és -szabályozó alrendszer fejlesztése LEO pályán működő kisműholdakhoz}}
\cventry{2019.01--02.}{FPGA Developer Intern}{Karlsruhe Institute of Technology}{Karlsruhe, Germany}{}{\en{Implemented FPGA time synchronisation with a Python interface}{FPGA időszinkronizáció implementálása Python interfésszel}}
\cventry{2018.06--08.}{Image Processing Intern}{Fraunhofer Institute for Factory Operation and Automation IFF}{Magdeburg, Germany}{}{\en{Developed an automated visual inspection tool in C++, including a Python wrapper}{Automatizált vizuális inspekciós szokftver fejlesztése C++-ban Pyhton interfésszel}}
\cventry{2017.06--08.}{Data Scientist Intern}{Gravity R\&D}{Budapest, \en{Hungary}{Magyarország}}{}{\en{Analyzed customer data in Python with machine learning}{Vásárlói adatok elemzése machine learning módszerekkel}}{}
\cventry{2016.09--2019.01.}{Virtual Reality Peripheral Device Developer}{\en{Budapest University of Technology and Economics}{Budapesti Műszaki és Gazdaságtudományi Egyetem}}{Budapest, \en{Hungary}{Magyarország}}{}{\en{Developed the hardware and software for a 3D input device for virtual working environments}{Hardver- és szoftverfejlesztés virtuális munkakörnyezetekben alkalmazható háromdimenziós beviteli eszközhöz}}








%%%%%%%%%%%%%%%%%%%%%%%%%%%%%%%%%%%%%%
\section{\en{Honors and Awards}{Díjak és Kitüntetések}}
%%%%%%%%%%%%%%%%%%%%%%%%%%%%%%%%%%%%%%
\cventry{2023,2024}{Qualcomm Innovation Fellowship Europe}{Qualcomm}{\en{Finalist}{Döntő}}{}{}{}
\cventry{2023}{\en{$\mathbf{4^{\mathrm{th}}}$ Place at German University Rowing Championship}{ $\mathbf{4.}$ helyezés a Német Egyetemi Evezős Bajnokságban}}{\en{German Rowing Association}{Deutscher Ruderverband}}{\en{Coxed Men's Quad}{Kormányzott négypárevezős} $500\ \mathrm{m}$}{}{}{}
\cventry{2022}{NeurIPS Scholar Award}{}{}{}{}{}
\cventry{2021}{Pro Scientia \en{Gold Medal}{Aranyérem}}{\en{National Scientific Students’ Association Hungary}{Országos Tudományos Diákköri Tanács}}{top 0.03\%}{}{}{}{}
\cventry{2021}{\en{Hope Badge Special Award to The Most Promising Young Scientist}{Reménység Kitűző Különdíj}}{Pro Scientia \en{Gold Medalists' Association}{Aranyérmesek Társasága}}{top 0.003\%}{}{}{}{}
\cventry{2019, 2021}{\en{$\mathbf{1^{\mathrm{st}}}$ Prize at National Scientific Students’ Association Conference}{$\mathbf{1.}$ helyezés az Országos Tudományos Diákköri Konferencián}}{\en{National Scientific Students’ Association Hungary}{Országos Tudományos Diákköri Tanács}}{top 0.3\%}{}{\begin{itemize}
    \item Attention-based curiosity in multi-agent reinforcement learning environments (2021)
    \item Stochastic weight matrix-based regularization methods for deep neural networks (2019)
    \item Development of a 3D input device for virtual working environments (2019)
\end{itemize}}
\cventry{2017--2019}{\en{$\mathbf{1^{\mathrm{st}}}$ Prize at Scientific Students’ Association Conference}{{$\mathbf{1.}$ helyezés a Tudományos Diákköri Konferencián}}}{\en{Budapest University of Technology and Economics}{Budapesti Műszaki és Gazdaságtudományi Egyetem}}{top 0.3\%}{}{\begin{itemize}
    \item Attention-based curiosity in multi-agent reinforcement learning environments (2019)
    \item Development of an Attitude Determination and Control System for CubeSats on LEO orbits (2019)
    \item Stochastic weight matrix-based regularization methods for deep neural networks (2018)
    \item Development of a 3D input device for virtual working environments (2017)
\end{itemize}}  
\cventry{2018}{Nokia Bell Labs Scholarship for Deep Learning Research}{Nokia Bell Labs \en{Hungary}{Magyarország}}{}{}{}
\cventry{2016,2018}{\en{New National Excellence Program Research Grant}{Új Nemzeti Kiválóság Program Kutatási Ösztöndíj}}{\en{Ministry of Innovation and Technology Hungary}{Innovációs és Technológiai Minisztérium}}{top 0.3\%}{}{}
\cventry{2016--2018,2020}{\en{National Higher Education Scholarship}{Nemzeti Felsőoktatási Ösztöndíj}}{\en{Republic of Hungary}{Innovációs és Technológiai Minisztérium}}{top 0.8\%}{}{}{}



%%%%%%%%%%%%%%%%%%%%%%%%%%%%%%%%%%%%%%
% Publications
%%%%%%%%%%%%%%%%%%%%%%%%%%%%%%%%%%%%%%
\nocite{*}
\bibliographystyle{unsrt}
\bibliography{publications}

%%%%%%%%%%%%%%%%%%%%%%%%%%%%%%%%%%%%%%
\section{\en{Talks}{Előadások}}
%%%%%%%%%%%%%%%%%%%%%%%%%%%%%%%%%%%%%%
\cventry{2025.02.}{Causality and OOD generalization in Foundation Models
}{Bellairs Workshop on Causality, Bellairs Research Institute}{Barbados}{\color{customblue} Invited}{}
\cventry{2024.09.}{Identifiable Exchangeable Mechanisms}{Mila tea talk series, Mila}{Montréal, Canada}{\color{customblue} Invited}{}
\cventry{2024.07.}{Embrace the Gap: VAEs Perform Independent Mechanism Analysis}{Critical ML Lab, University of Waterloo}{Waterloo, Canada}{\color{customblue} Invited}{}
\cventry{2023.12.}{Embrace the Gap: VAEs Perform Independent Mechanism Analysis}{Central European University Representation Learning Reading Group}{Budapest, Hungary}{\color{customblue} Invited}{}
\cventry{2023.02.}{Popper meets machine learning---How falsificationism can guide the design of AI solutions}{Darwin College}{Cambridge, UK}{}{}
\cventry{2023.01.}{Multivariable Causal Discovery for General Nonlinear Functions}{AstraZeneca Seminar}{online, \color{customblue} Invited}{}{}
\cventry{2022.12.}{Multivariable Causal Discovery for General Nonlinear Functions}{Learning on Graphs Cambridge Meetup}{Cambridge, UK}{}{}
\cventry{2022.10.}{Embrace the Gap: VAEs Perform Independent Mechanism Analysis}{University of Warsaw Machine Learning Seminar}{online, \color{customblue} Invited}{}{}
\cventry{2022.08.}{Multivariable Causal Discovery for General Nonlinear Functions}{UAI 2022 Workshop on Causal Representation Learning}{Eindhoven, \en{Netherlands}{Hollandia}}{}{}


%%%%%%%%%%%%%%%%%%%%%%%%%%%%%%%%%%%%%%
\section{\en{Teaching}{Oktatás}}
%%%%%%%%%%%%%%%%%%%%%%%%%%%%%%%%%%%%%%
\cventry{\en{Fall}{} 2020/21\en{}{/1}}{\en{Teaching Assistant}{Demonstrátor}}{\en{Budapest University of Technology and Economics}{Budapesti Műszaki és Gazdaságtudományi Egyetem}}{Budapest, \en{Hungary}{Magyarország}}{}{ \en{Image Processing Laboratory}{Képfeldolgozás Laboratórium} I., \en{Computer Vision Systems}{Számítógépes Látórendszerek}, Deep Learning \en{in Visual Computing}{Alkalmazása a Vizuális Informatikában}}
\cventry{\en{Fall}{} 2017/18\en{}{/1}}{\en{Teaching Assistant}{Demonstrátor}}{\en{Budapest University of Technology and Economics}{Budapesti Műszaki és Gazdaságtudományi Egyetem}}{Budapest, \en{Hungary}{Magyarország}}{}{\en{Digital Design}{Digitális Technika} I. \en{laboratory}{labor}}

%%%%%%%%%%%%%%%%%%%%%%%%%%%%%%%%%%%%%%
\section{\en{Extracurriculars \& Coursework}{Kiegészítő Tanulmányok}}
%%%%%%%%%%%%%%%%%%%%%%%%%%%%%%%%%%%%%%
\cventry{2023.12.}{CI/CD for Machine Learning cerfification}{Weights and Biases}{online}{}{}
\cventry{2022.07.}{ELLIS Cambridge Unit Machine Learning Summer School}{ELLIS Cambridge Unit}{Cambridge, UK}{}{}
\cventry{2022.07.}{Machine Learning Summer School}{ML in PL}{\en{Krakow}{Krakkó}, \en{Poland}{Lengyelország}}{}{}
\cventry{2021.04.}{A Young Leader’s Guide to Risk}{McChrystal Group}{Budapest, \en{Hungary}{Magyarország}}{}{}
\cventry{2020.09.}{Ladybird Guide to Spacecraft Operations Workshop}{European Space Agency}{online}{}{}
\cventry{2020.07.}{Eastern European Machine Learning Summer School}{ML in PL}{online}{}{}
\cventry{2019.07.}{International Summer School on Deep Learning}{IRDTA}{\en{Warsaw}{Varsó}, \en{Poland}{Lengyelország}}{}{}
\cventry{2019.01.}{Concurrent Design Workshop}{European Space Agency}{ESEC-Galaxia, Redu, Belgium}{}{}
\cventry{2018--2020}{Leadership Academy}{Mathias Corvinus Collegium}{Budapest, \en{Hungary}{Magyarország}}{}{}
\cventry{2018.11.}{Traction Europe Case Studies for Outstanding Engineering Students}{Boston Consulting Group}{\en{Paris}{Párizs}, \en{France}{Franciaország}}{}{}
\cventry{2016--2018}{\en{Business and Economics Specialization}{Közgazdaságtan és Business Szakirány}}{Mathias Corvinus Collegium}{Budapest, \en{Hungary}{Magyarország}}{}{}
\cventry{2015--2016}{\en{University Junior Program}{Egyetmi Program Juniorképzés}}{Mathias Corvinus Collegium}{Budapest, \en{Hungary}{Magyarország}}{}{}

% \newpage



%%%%%%%%%%%%%%%%%%%%%%%%%%%%%%%%%%%%%%
\section{Outreach \& Community Service}
%%%%%%%%%%%%%%%%%%%%%%%%%%%%%%%%%%%%%%
\cventry{2024}{Organizer}{CALM: First Workshop on
Causality and Large Models}{NeurIPS 2024}{}{}
\cventry{2021--ongoing}{\en{Thesis committee member}{Szakdolgozat Bíráló}}{\en{Budapest University of Technology and Economics}{Budapesti Műszaki és Gazdaságtudományi Egyetem}}{}{}{}
\cventry{2020--ongoing}{\en{Coordinator}{Koordinátor}}{\en{}{Magyar} Machine Learning Journal Club \en{for Hungarian Students}{}}{}{}{}
\cventry{2020}{Program Committee Member}{6th International Conference on Machine Learning, Optimization, and Data Science}{}{}{}
\cventry{2016}{E-learning Developer}{EduBase}{}{}{\en{Video series for the Digital Design I. university course}{Videósorozat készítése a Digitális Technika I. egyetemi kurzushoz}}


%%%%%%%%%%%%%%%%%%%%%%%%%%%%%%%%%%%%%%
\section{Mentoring}
%%%%%%%%%%%%%%%%%%%%%%%%%%%%%%%%%%%%%%
% \cventry{2024--ongoing}{Simon Hanrath}{M.Sc. at the University of Tübingen}{}{}{}
\cventry{2025--ongoing}{Samuel Innes}{B.Sc. at the University of Heidelberg}{}{}{}
\cventry{2023--ongoing}{Bálint Mucsányi}{M.Sc. at the University of Tübingen$\to$ Ph.D. at MPI-IS}{}{}{}
\cventry{2023--ongoing}{Szilvia Ujváry}{M.Sc.$\to$ Ph.D. at the University of Cambridge}{}{}{}
\cventry{2023--ongoing}{Anna Mészáros}{M.Sc.$\to$ Ph.D. at the University of Cambridge}{}{}{}
\cventry{2022--ongoing}{Boglárka Ecsedi}{B.Sc. at GeorgiaTech}{}{}{}

%%%%%%%%%%%%%%%%%%%%%%%%%%%%%%%%%%%%%%
\section{\en{Reviewing}{Bíráló}}
%%%%%%%%%%%%%%%%%%%%%%%%%%%%%%%%%%%%%%
\cventry{}{\mdseries ICML 2025 position paper track, ICLR 2025, NeurIPS 2024, CLeaR 2024/2025, NeurIPS 2023/2024 workshops, Infocommunications Journal}{}{}{}{}


%%%%%%%%%%%%%%%%%%%%%%%%%%%%%%%%%%%%%%
\section{\en{Competencies}{Kompetenciák}}
%%%%%%%%%%%%%%%%%%%%%%%%%%%%%%%%%%%%%%
% \cvskillhead[-0.1em][Level][Skill][Years][]%
% \cvskillentry*{Programming}{5}{Python}{5}
% % \cvskillentry*{}{3}{C++}{2}
% % \cvskillentry{}{3}{\LaTeX}{14}
% \cvskillentry*{OS:}{3}{Linux}{2}
% % \cvskillentry*[1em]{Methods}{4}{SCRUM}{8}

\cvitem{Machine Learning}{\begin{itemize}
\item[] \cvskill{5} PyTorch
\item[] \cvskill{5} PyTorch Lightning
\item[] \cvskill{5} Weights and Biases
\end{itemize}
}

\cvitem{Software Engineering}{\begin{itemize}
\item[] \cvskill{5} Git
\item[] \cvskill{4} CI/CD
\item[] \cvskill{3} Singularity
\end{itemize}
}

\cvitem{\en{Programming Languages}{Programozási Nyelvek}}{\begin{itemize}
    \item[] \cvskill{5} Python
    \item[] \cvskill{4} C++
    \item[] \cvskill{3} C
\end{itemize}
}
                
\cvitem{\en{Research}{Kutatás}}{\begin{itemize}
\item[] \cvskill{5} Zotero
\item[] \cvskill{4} \LaTeX
\end{itemize}
}


%%%%%%%%%%%%%%%%%%%%%%%%%%%%%%%%%%%%%%
\section{\en{Languages}{Nyelvismeret}}
%%%%%%%%%%%%%%%%%%%%%%%%%%%%%%%%%%%%%%
\cvitem{\en{English}{Angol}}{\en{Proficient}{C1}}
\cvitem{\en{German}{Német}}{\en{Proficient}{C1}}
\cvitem{\en{Italian}{Olasz}}{\en{Elementary}{A1}}
\cvitem{\en{Hungarian}{Magyar}}{\en{Native}{Anyanyelv}}





\vfill
% \vspace{40pt}
Tübingen, \en{\today}{\hundate\today}



\end{document}

